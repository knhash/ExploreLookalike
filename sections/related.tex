\section{Related Work}

Most of the work in recommendation systems has been around collaborative filtering-based approaches \cite{zhang2014huiyi}. But in practice, the effectiveness of collaborative filtering models is limited by the sparsity of interaction matrix of historical users and cold start users \cite{ahn2008new}. The sparsity is majorly because users show explicit interest in only a few pieces of content. With the increase in time duration, the sparsity increases even more. Many have attempted to overcome this limitation by adopting different approaches such as clustering or dimensionality reduction, whose key motive was to use low-dimensional matrix instead of original high dimensional matrix. Instances of these kind of approaches can be seen in \cite{cheng2000biclustering} \cite{barragans2010hybrid} \cite{luo2014efficient} which primarily used bicluster algorithms, singular value decomposition and non-negative matrix factorization.  

Working on the same lines, authors in \cite{jin2020music} combined the latent-factor approach and the clustering approach to obtain a finer representation of their interaction data. Latent factorization helped to break the user-item matrix into user preference matrix and item preference matrix, on which clustering is done individually. A collaborative filtering model is then plugged in for predictions.  

In \cite{das2014clustering}, the authors adopted a slightly different approach, wherein, a DBSCAN clustering algorithm is employed to cluster users. Users are served items based on different voting algorithms for each cluster separately. The users in specific clusters are recommended items based on the ratings of the users of that cluster only.  

A similar work is proposed by the authors in \cite{brodinova2019robust}, where they used K-Means clustering along with a weighing function which assigned weights to each of the observations. They further used the lasso-type penalty in their objective function to reduce the dimensionality of their data. 